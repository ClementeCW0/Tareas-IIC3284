\begin{problema}{1}
	\begin{enumerate}[label=\arabic*.]
		\item El siguiente enfoque se llama reservoir sampling. Supongamos que tenemos una secuencia
de elementos pasando uno a la vez. Queremos mantener una muestra de un elemento con
la propiedad de que está distribuida uniformemente sobre todos los elementos que hemos
visto en cada paso. Además, queremos lograr esto sin conocer el número total de elementos
de antemano o almacenar todos los elementos que vemos. \\
\\
Considere el siguiente algoritmo, que almacena solo un elemento en memoria en todo
momento. Cuando aparece el primer elemento, se almacena en la memoria. Cuando aparece
el $k$-ésimo elemento, reemplaza el elemento en memoria con probabilidad $1/k$. Explique
por qué este algoritmo resuelve el problema.
\item Sea un modelo de $m$ ítems y $n$ urnas, asignamos de forma independiente y con probabilidad uniforme $1/n$ los ítems en las urnas. Denotamos la carga de una urna $U_i$ para $i \in [n]$ como la cantidad de ítems que contiene la urna $i$ luego de la asignación de ítems.
	\begin{enumerate}[label=\alph*)]
		\item Calcule el valor esperado de la carga para una urna $U_i$ con $i \in [n]$.
		\item Calcule la varianza de la carga para una urna $U_i$ con $i \in [n]$.
		\item Para un $a > 0$, use la desigualdad de Chebishev para acotar
			\[
				\P[|U_i - \E[U_i]| \geq a]
			\]
		
	\end{enumerate}
	\end{enumerate}
\end{problema}
\begin{solucion}

\end{solucion}

\newpage
\begin{problema}{2}
	\begin{enumerate}[label=\arabic*.]
		\item Una permutación de los números $[n] = \{1,\dots, n\}$ puede ser representada como una función $\pi : [n] \to [n]$, donde $π(i)$ es la posición de $i$ en el orden dado por la permutación. Un punto fijo de una permutación $\pi : [n] \to [n]$ es un valor para el cual $\pi(x) = x$. Encuentre el número esperado y la varianza de puntos fijos de una permutación elegida uniformemente al azar entre todas las permutaciones.
		\item Dada una permutación aleatoria de ${1, \dots , n}, n \in \N$, sumamos los números que aparecen como \emph{left-to-right minima}. ¿Cuál es la suma esperada?
	\end{enumerate}
\end{problema}
\begin{solucion}
	
\end{solucion}

\newpage
\begin{problema}{3}
	Supongamos que lanzamos una moneda justa $n$ veces para obtener $n$ bits aleatorios. Consideremos todos los $m = \binom{n}{2}$ pares de estos bits en algún orden. Sea $Y_i$ la disyunción exclusiva (\emph{xor}) del $i$-ésimo par de bits, y definamos $Y = \sum^m_{i=1} Y_i$ como el número de $Y_i$ que son iguales a 1. \emph{Disyunción exclusiva: Dados dos bits a, b ∈ {0, 1}, definimos la disyunción exclusiva (xor) como}
	\[
		a \oplus b = \begin{cases}
			0 & \text{\emph{ si $a=b$}} \\
			1 & \text{\emph{ si $a\neq b}}
		\end{cases}
	\]
	\emph{Es decir, el resultado es 1 únicamente cuando exactamente uno de los dos bits vale 1.}
	\begin{enumerate}[label=\alph*)]
		\item Muestre que cada $Y_i$ es $0$ con probabilidad $1/2$ y $1$ con probabilidad $1/2$.
		\item Muestre que los $Y_i$ no son mutuamente independientes.
		\item Muestre que los $Y_i$ satisfacen la propiedad $\E[Y_iY_j]=\E[Y_i]\E[Y_j]$.
		\item Calcule Var[$Y$]. \emph{Use que si $\forall 1 \leq i < j \leq n \ Y_i, Y_j
	\end{enumerate}
\end{problema}
